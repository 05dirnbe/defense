


\listfiles

\documentclass[	hyperref={pdfpagelabels=false}, xcolor=dvipsnames,
		11pt]{beamer}


\newcommand*{\PTAS}{PTAS}

\newcommand*{\IE}{i.e.}
\newcommand*{\EG}{e.g.}

\newcommand*{\keyword}[1]{\emph{#1}\index{#1}}
% \newcommand*{\comments}[1]{}

\newcommand{\CORR}{\mathrel{\widehat{=}}}

\newcommand*{\SU}[1]{SU(#1)}
% \newcommand*{\Z}[1]{Z(#1)}
\newcommand*{\Z}[1]{\mathbb{Z}_{#1}}

\newcommand*{\DIM}[1]{#1-dimensional}
\newcommand*{\COMMA}{\quad ,}
\newcommand*{\POINT}{\quad .}
\newcommand*{\MEAS}[1]{\mathcal{D}[#1]}


\DeclareMathOperator{\Tr}{tr}
\DeclareMathOperator{\e}{e}

\newcommand*{\tr}[1]{\Tr \ [ \ #1 \ ]}

\newcommand*{\FIG}[1]{Figure~\ref{#1}}
\newcommand*{\FIGs}[2]{Figures~\ref{#1} - \ref{#2}}
\newcommand*{\TAB}[1]{Table~\ref{#1}}
\newcommand*{\SEC}[1]{Section~\ref{#1}}
\newcommand*{\CHAP}[1]{Chapter~\ref{#1}}
\newcommand*{\LIST}[1]{Listing~\ref{#1}}
\newcommand*{\PROC}[1]{Procedure~\ref{#1}}
\newcommand*{\EXP}[1]{\langle #1 \rangle}
\newcommand*{\EQU}[1]{Equation~\eqref{#1}}
\newcommand*{\EQUs}[2]{Equations~(\ref{#1}) - (\ref{#2})}
\newcommand*{\APP}[1]{Appendix~\ref{#1}}

%\newcommand{\tr}[1]{\mbox{tr}~\left[#1]}

\renewcommand\Re{\operatorname{Re}}
\renewcommand\Im{\operatorname{Im}}




\newcommand*{\SYM}[1]{\mathcal{#1}}

\newcommand*{\FIGURE}[1]{\textcolor{blue}{Fig.:}}
% \newcommand*{\FIGURE}[1]{}

%  \newcommand{\testbox}[1]{\framebox{#1}}
\newcommand{\testbox}[1]{#1}

\setlength{\fboxsep}{0.0pt}	%	separation between boxes.
\setlength{\fboxrule}{0.05pt}	%	line width of boxes

\newlength{\plotwidth}
\setlength{\plotwidth}{360pt}

\newlength{\testwidth}
\setlength{\testwidth}{0.25\textwidth}
 
\setlength{\unitlength}{0.95\textwidth}  % measure in textwidths




% \def\dblone{\hbox{$1\hskip -1.2pt\vrule depth 0pt height 1.6ex width 0.7pt\vrule depth 0pt height 0.3pt width 0.12em$}}

%     \setbeamerfont{section title}{parent=title}
%     \setbeamercolor{section title}{parent=titlelike}
%     \defbeamertemplate*{section page}{default}[1][]
%     {
%       \centering
%         \begin{beamercolorbox}[sep=8pt,center,#1]{section title}
%           \usebeamerfont{section title}\insertsection\par
%         \end{beamercolorbox}
%     }
%     \newcommand*{\sectionpage}{\usebeamertemplate*{section page}}

\definecolor{arrowred}{RGB}{255,105,105}
\definecolor{arrowblue}{RGB}{105,105,255}
\definecolor{resultgreen}{RGB}{246,255,213}
\definecolor{resultgreenedge}{RGB}{0,128,0}


\AtBeginSection[]{
  \begin{frame}
  \vfill
  \centering
  \begin{beamercolorbox}[sep=8pt,center,shadow=true,rounded=true]{title}
    \usebeamerfont{title}\insertsectionhead\par%
  \end{beamercolorbox}
  \vfill
  \end{frame}
}




\usepackage{etex}			% needed to resolve conflicts between booktabs and tikz
\usepackage{tikz}
\usepackage{verbatim}
\usetikzlibrary{arrows,shapes}

 %\usetheme{boxes}
% \usecolortheme{lily}
 %\usecolortheme{rose}

 \usefonttheme{serif}

\usepackage{lmodern}
 \usepackage{graphicx}
 \usepackage{multirow}				%enables multirow tables

\usepackage{natbib}
\bibpunct{[}{]}{,}

\usepackage{epsfig}
\usepackage{layout}

\usepackage{booktabs}

\setbeamercovered{transparent}
\mode<presentation>





 
\beamertemplatenavigationsymbolsempty %Navigationszeile auf jeder Folie unterdrücken






\usepackage{ae,aecompl}				% fuer besseres pdf empfohlen
%\usepackage{exscale}				% richtige Skalierung der Mathe Formeln. Does not work with tikz




\usepackage[mathcal]{euscript}
\usepackage{amsthm}				% adds theorems and lemmata



\usepackage{dsfont}

% \usepackage{fancyhdr}	


% 
% %******************************************************************************
% %
% % Fancyheaders
% %
% \usepackage{fancyhdr}					% Pagehead
% \pagestyle{fancy}					% meine Kopfzeile
% \fancyhf{}
% % \fancyhead[RO]{\rm \nouppercase  \rightmark \qquad \rm \thepage }
% % \fancyhead[LE]{  \thepage  \qquad \rm \nouppercase  \leftmark }
% \fancyhead[RO]{\nouppercase  \rightmark}
% \fancyhead[LE]{ \nouppercase  \leftmark }
% 
% \fancyfoot[OR]{  \thepage}
% \fancyfoot[EL]{ \thepage }
% 
% % \renewcommand{\headrulewidth}{0.1pt}
% 
% \renewcommand{\chaptermark}[1]{\markboth{\thechapter.\ #1}{}}
% 
% 
% \renewcommand{\sectionmark}[1]{\markright{\thesection.\ #1}}
% 
% 
% \fancypagestyle{plain}{%
% \fancyhf{} % clear all header and footer fields
% \fancyfoot[RO,RE]{\thepage} %RO=right odd, RE=right even
% \renewcommand{\headrulewidth}{0pt}
% \renewcommand{\footrulewidth}{0pt}}
% 
% 
% 
% %
% %******************************************************************************






\usepackage{array}		%for readjusting the hight of lines in tables
% \usepackage{longtable}

\usepackage[german,english]{babel}
\usepackage{hyperref}
\hypersetup{colorlinks=false}
%
\DeclareGraphicsExtensions{.eps, .jpg, .png, .sgv, .jepg}
% \usepackage[subsection]{algorithm}
% \usepackage{algorithmic,algorithmic-fix}
\usepackage{listings}
\lstset{ %
language=C++,                   % the language of the code
basicstyle=\footnotesize,       % the size of the fonts that are used for the code
numbers=left,                   % where to put the line-numbers
numberstyle=\footnotesize,      % the size of the fonts that are used for the line-numbers
stepnumber=1,                   % the step between two line-numbers. If it's 1, each line 
                                % will be numbered
numbersep=5pt,                  % how far the line-numbers are from the code
% showspaces=false,               % show spaces adding particular underscores
% showstringspaces=false,         % underline spaces within strings
showtabs=false,                 % show tabs within strings adding particular underscores
frame=bottomline,                   % adds a frame around the code
% tabsize=2,                      % sets default tabsize to 2 spaces
captionpos=t,                   % sets the caption-position to bottom
% breaklines=false,                % sets automatic line breaking
% breakatwhitespace=false,        % sets if automatic breaks should only happen at whitespace
% title=\lstname,                 % show the filename of files included with \lstinputlisting;
                                % also try caption instead of title
% escapeinside={\%*}{*)},         % if you want to add a comment within your code
% morekeywords={*,...}            % if you want to add more keywords to the set
}


\usepackage{caption}
\DeclareCaptionFont{white}{\color{white}}
\DeclareCaptionFormat{listing}{\colorbox{gray}{\parbox{\textwidth}{#1#2#3}}}
\captionsetup[lstlisting]{format=listing,labelfont=white,textfont=white}











\usepackage{parskip}		%kills the default paragraph indenting and sets /parskip to some useful value automatically



\usepackage{amssymb}
\usepackage{amsmath}
% \usepackage{bbold}
% \usepackage{dsfont}

%Graphics and Videos




%Graphics and Videos

% \usepackage{movie15}





 
\logo{\includegraphics[width=1cm]{unilogo}}


\title{Physarum Computing:\\ Slime on Shortest Paths}
\author{Michael T. Dirnberger}
\institute{Max Planck Institute for Informatics Saarbr\"ucken}
%\date{}

\date{Graph Theory, Algorithms and Applications, Erice, 2014\\[2em]
% {\footnotesize Acknowledgement: This work is}
% 
}


\begin{document}





\begin{frame}[plain]

\titlepage
\vspace{-1cm}
	    \begin{center}
		\includegraphics[width=0.3\linewidth]{./pics/mpilogo.jpg}
	    \end{center}
\end{frame} 

% \begin{frame}
%     \frametitle{Outline}
%     \tableofcontents
% \end{frame} 


%%%%%%%%%%%%%%%%%%%%%%%%%%%%%%%%%%%%%%%%%%%%%%%%%%%%%%%%%%%%%%%%%%%%%%%%%%%%%%%%%%%%%%%%%%%%%%%%%%%%%%%%%%%%%%%%%%%%%%%%%%%%%%%%%%%%%%%%%%%%%%%%%%%%%%
\section{Physarum Computing} 
%%%%%%%%%%%%%%%%%%%%%%%%%%%%%%%%%%%%%%%%%%%%%%%%%%%%%%%%%%%%%%%%%%%%%%%%%%%%%%%%%%%%%%%%%%%%%%%%%%%%%%%%%%%%%%%%%%%%%%%%%%%%%%%%%%%%%%%%%%%%%%%%%%%%%%
\subsection{Physarum Polycephalum}
%%%%%%%%%%%%%%%%%%%%%%%%%%%%%%%%%%%%%%%%%%%%%%%%%%%%%%%%%%%%%%%%%%%%%%%%%%%%%%%%%%%%%%%%%%%%%%%%%%%%%%%%%%%%%%%%%%%%%%%%%%%%%%%%%%%%%%%%%%%%%%%%%%%%%%

\begin{frame}
    \frametitle{True slime molds in a nutshell} 


 %\vspace{0cm}

\begin{columns}
\begin{column}{4.6cm}

\begin{overprint}

  
	\begin{block}{Physarum Polycephalum:}
	  \begin{itemize}
	   \item<2> Unicellular organism with many nuclei
	   \item<3> Intricate foraging strategy
	   \item<4> Networks distribute protoplasm
	   
	  \end{itemize}
	\end{block}


\end{overprint}

\end{column}

\begin{column}{5cm}
\begin{overprint}


\testbox{
     \begin{minipage}[t]{5 cm}

\begin{figure}[h]
 
     \begin{center}
      \testbox{\includegraphics[angle=0,clip=true,width= 1.6\onethird, trim = 0 0 0 0]<1>{./pics/tree_of_life.png}}
      \testbox{\includegraphics[angle=0,clip=true,width= 1.6\onethird, trim = 0 0 0 0]<2>{./pics/physarum_forest.jpg}}
      \testbox{\includegraphics[angle=0,clip=true,width= 1.6\onethird, trim = 0 0 0 0]<3>{./pics/physarum_exploring_tree_2.jpg}}
      \testbox{\includegraphics[angle=0,clip=true,width= 1.5\onethird, trim = 0 0 0 0]<4>{./pics/physarum.jpg}}
     \end{center}

\end{figure}
     \end{minipage} }

\testbox{
     \begin{minipage}{5 cm}
      \begin{center}

	\tiny{Images courtesy Prof. Tetsuo Ueda, Sapporo University, Japan.}
	

      \end{center}
     \end{minipage} }






\end{overprint}
\end{column}
\end{columns}

\vspace{-1cm}

\centering
	\begin{alertblock}{\underline{Intelligent behavior:}}
	  Physarum can adapt to various environmental conditions very efficiently. This hints at an underlying optimization process.
	\end{alertblock}



\end{frame}


%%%%%%%%%%%%%%%%%%%%%%%%%%%%%%%%%%%%%%%%%%%%%%%%%%%%%%%%%%%%%%%%%%%%%%%%%%%%%%%%%%%%%%%%%%%%%%%%%%%%%%%%%%%%%%%%%%%%%%%%%%%%%%%%%%%%%%%%%%%%%%%%%%%%%%

\subsection{Modelling Physarum}

\begin{frame}
       \frametitle{A basic feed-back mechanism describing Physarum} 
   
       \begin{figure}
        
      

	    \includegraphics[angle=0,clip=true,width=\textwidth, trim = 0 0 0 0]{./pics/model_new.pdf}
	 
	    \vspace{1cm}
	    
      \begin{center}
      \caption{Veins/edges carrying a large amount of flow (red) are \alert{{\bf reinforced}} while all others (black) are 
	      \textcolor{arrowblue}{{\bf minimized}}.}
      \end{center}

      \end{figure}


\end{frame}


%%%%%%%%%%%%%%%%%%%%%%%%%%%%%%%%%%%%%%%%%%%%%%%%%%%%%%%%%%%%%%%%%%%%%%%%%%%%%%%%%%%%%%%%%%%%%%%%%%%%%%%%%%%%%%%%%%%%%%%%%%%%%%%%%%%%%%%%%%%%%%%%%%%%%%


\begin{frame}
    \frametitle{A dynamic evolution process {\footnotesize(Ito, Johannson, Nagagaki, Tero, 2011)}} 

	Let $G = (V,E)$ be a graph. For every edge $e \in E$ define:
	     
	\begin{itemize}
	  \item $x_e(t)$ as the time-dependent capacity. 
	  \item $l_e$ as the constant edge-length. 
	  \item $a_e$ as the constant edge-reactivity. 
	  \item $r(t) = a_e l_e / x_e(t)$ as the time-dependent resistance.
	\end{itemize}  

        In addition, let $q_e(r_e(t))$ denote the electric flow over an edge $e$. 

	  \begin{alertblock}{\underline{Evolution equation for $x_e(t)$:}}
	  
	  \vspace{-0.5cm}
	  \begin{equation}  
	    \dot{x}_e(t) = |q_e(r(t))| - a_e x_e(t)  \nonumber
	  \end{equation}
	  \end{alertblock}
 
    


\end{frame}

%%%%%%%%%%%%%%%%%%%%%%%%%%%%%%%%%%%%%%%%%%%%%%%%%%%%%%%%%%%%%%%%%%%%%%%%%%%%%%%%%%%%%%%%%%%%%%%%%%%%%%%%%%%%%%%%%%%%%%%%%%%%%%%%%%%%%%%%%%%%%%%%%%%%%%


\begin{frame}
    \frametitle{An electric analogy} 

	Let $p_u(t)$ be the electric potential at node $u$ at time $t$. 
	
	\begin{alertblock}{\underline{Flow $q_e$ on edge $e = \{u,v\}$:} \textcolor{Gray}{(Ohm's Law)} }
	  \begin{center}
	  \begin{tabular}{ccc}
	 

	 
	    $q_e = (\frac{a_e l_e}{  x_e})^{-1} (p_u - p_v )$ & $\quad $ & \textcolor{Gray}{$I = R^{-1} U $} \\
	 
	 
	  \end{tabular}
	  \end{center}
	\end{alertblock}

	
	\begin{alertblock}{\underline{Flow conservation:} \textcolor{Gray}{(Kirchhoff's Current Law)}}
	  \begin{center}
	  \begin{tabular}{ccc}
	 

	 
	    $\sum\limits_{e = \{u,v\} \in E} q_e = b_u $ & $\quad $ & \textcolor{Gray}{$\sum\limits_{k} I_k = 0$} \\
	 
	 
	  \end{tabular}
	  \end{center}
	\end{alertblock}

	Setting $b_s = - b_t = 1$ as well as $ b_u = 0$ amounts to sending one unit of flow/current from source node $s$ to sink node $t$.




\end{frame}


%%%%%%%%%%%%%%%%%%%%%%%%%%%%%%%%%%%%%%%%%%%%%%%%%%%%%%%%%%%%%%%%%%%%%%%%%%%%%%%%%%%%%%%%%%%%%%%%%%%%%%%%%%%%%%%%%%%%%%%%%%%%%%%%%%%%%%%%%%%%%%%%%%%%%%

\subsection{A minimal example}

\begin{frame}
    \frametitle{A minimal example ($a_1 = a_2 = 1$)} 
      
      \tikzstyle{vertex}=[circle,fill=black!25,minimum size=15pt,inner sep=0pt]
      \tikzstyle{edge} = [draw,thick,-]
      \tikzstyle{weight} = [font=\small]
     

\begin{figure}[ht]
\begin{minipage}[b]{0.45\linewidth}
\centering
	  \begin{equation}  
	    \dot{x}_e = |q_e| - x_e  \nonumber
	  \end{equation}

\label{fig:figure1}
\end{minipage}
\hspace{0.5cm}
\begin{minipage}[b]{0.45\linewidth}
\centering
 \begin{tikzpicture}[scale=0.8, auto,swap]
      % draw the vertices
      \foreach \pos/\name in {{(0,2)/s}, {(3,2)/t}}
	  \node[vertex] (\name) at \pos {$\name$};
      % Connect vertices with edges and draw weights
    
      
      \foreach \source/ \dest /\weight in {s/t/l_1}
	  \path[edge] (\source) to[bend left] node[weight] {$\weight$} (\dest);
      
      \foreach \source/ \dest /\weight in {s/t/l_2}
	  \path[edge] (\source) to[out=-80,in=-100] node[weight] {$\weight$} (\dest);
      

      
      \end{tikzpicture}

\label{fig:figure2}
\end{minipage}
\end{figure}

 \vspace{-0.75cm}

      \begin{equation}
	\frac{d}{dt} (x_1 + x_2) = |\underbrace{q_1 + q_2}_{1}| - (x_1 + x_2) = 1 - (x_1 + x_2) \nonumber
      \end{equation}

     
  
       Condition \alert{$\dot{x} = 0$ holds for equilibrium points}, thus $(x_1+x_2) \to 1$.
       
%        Let $x_1(0) = x_2(0) = \frac{1}{2}$ and $l_1 < l_2$.


 \vspace{-0.75cm}     
  
      \begin{equation}
	\left.\begin{aligned}
 	 l_1 & < l_2 \quad \text{and} \quad x_1(0)  = x_2(0) = \frac{1}{2} \\
        r_e & = \frac{l_e}{x_e}& \Rightarrow \quad r_1 < r_2 \quad \\
        q_e & = r_e^{-1} (p_s - p_t )& \Rightarrow \quad q_1 > q_2 \quad
       \end{aligned}
	\right\}
	\quad q_1 > \frac{1}{2} > q_2 \nonumber
       \end{equation}

      Thus $x_1 \to 1$ and simultaneously $x_2 \to 0$ for large $t$.






     

      

\end{frame}

%%%%%%%%%%%%%%%%%%%%%%%%%%%%%%%%%%%%%%%%%%%%%%%%%%%%%%%%%%%%%%%%%%%%%%%%%%%%%%%%%%%%%%%%%%%%%%%%%%%%%%%%%%%%%%%%%%%%%%%%%%%%%%%%%%%%%%%%%%%%%%%%%%%%%%

\subsection{Overview of results}

\begin{frame}
    \frametitle{Overview of convergence results} 

     \begin{center}
	  
	 

	  \begin{tabular}{ll}
% 	   \toprule  
	   Continuous:		& $\dot{x}_e(t) = f(q_e) - a_e x_e(t)$ \\
	   Discretized: & $x_e(t+h) = x_e(t) + h* \cdot [ f(q_e) - a_e x_e(t) ] + O(h^2) $ \\ 
% 	  \bottomrule
	  \end{tabular}
  
	  \vspace{0.5cm}

	  \begin{tabular}{cccc}
	  \toprule
	
	
		Model	  & $f(q_e)$	 	& $a_e = 1$ & $ a_e >0$ \\
	  \midrule
	  Continuous      & $|q_e|$     &  Bonifaci et al. (2012) 	& \alert{Open}   \\
			  & $q_e$       & Ito et al. (2011) 		& \textbf{\textcolor{OliveGreen}{Proof}} \\
	  Discretized     & $|q_e|$     & \textbf{\textcolor{OliveGreen}{$(1+\epsilon)$-approx.}}  & \textbf{\textcolor{OliveGreen}{$(1+\epsilon)$-approx.}}\\
			  & $q_e$	&  \textbf{\textcolor{OliveGreen}{Proof}}  & \alert{Open}\\		\\
	  \bottomrule
	  \end{tabular}

      \textcolor{OliveGreen}{\footnotesize{Joined work with: L. Becchetti, V. Bonifaci, A. Karrenbauer, K. Mehlhorn (ICALP 2013)}}
	  

      \end{center}
\end{frame}



%%%%%%%%%%%%%%%%%%%%%%%%%%%%%%%%%%%%%%%%%%%%%%%%%%%%%%%%%%%%%%%%%%%%%%%%%%%%%%%%%%%%%%%%%%%%%%%%%%%%%%%%%%%%%%%%%%%%%%%%%%%%%%%%%%%%%%%%%%%%%%%%%%%%%%


%%%%%%%%%%%%%%%%%%%%%%%%%%%%%%%%%%%%%%%%%%%%%%%%%%%%%%%%%%%%%%%%%%%%%%%%%%%%%%%%%%%%%%%%%%%%%%%%%%%%%%%%%%%%%%%%%%%%%%%%%%%%%%%%%%%%%%%%%%%%%%%%%%%%%%

\subsection{Sketch of proof}

\begin{frame}
    \frametitle{Distinct equilibria correspond to distinct paths} 

	\begin{block}{\underline{Assumptions:}}
	 \begin{enumerate}
	 \item Distinct $s-t$ paths have distinct lengths
	 \item $x(0) > 0$
	 \item $\textbf{1}^{S}_{T} \cdot x(0) \ge \textbf{1}^{T}_{0} \cdot x(0) = 1 $
	\end{enumerate}
	\end{block}

	


	\begin{alertblock}{\underline{Lemma:} }
	An equilibrium point of the evolution equation is a vector $x \in \mathbb{R}_{+}^{E}$ such that $\dot x = 0$. By assumption there are a finite
        number of such points. Every equilibrium point corresponds to a distinct s-t path and vice versa.
	\end{alertblock}


	

	

\end{frame}

%%%%%%%%%%%%%%%%%%%%%%%%%%%%%%%%%%%%%%%%%%%%%%%%%%%%%%%%%%%%%%%%%%%%%%%%%%%%%%%%%%%%%%%%%%%%%%%%%%%%%%%%%%%%%%%%%%%%%%%%%%%%%%%%%%%%%%%%%%%%%%%%%%%%%%



\begin{frame}
    \frametitle{ Finding equilibria via a Lyapunov function } 

	
	\begin{alertblock}{\underline{Goal:} Track energy dissipation $\eta$ s.t. flow conservation}
	 \center
	  $\eta =  \sum\limits_{e  \in E} r_e q_e^2 = p_s - p_t = q^T R q$ 

	\end{alertblock}



	\begin{block}{Lyapunov function $V$:}
	  \center
	  $V = l^{T} x = \sum\limits_{e  \in E} l_e x_e = \sum\limits_{e  \in E} r_e x_e^2 = x^T R x$ 
	 
	\end{block}

	\begin{block}{Properties of $V$:}
	  \center
	 \begin{tabular}{llll}
% 	   \toprule  
	   Properties:		& 1.) $V \ge 0$ & 2.) $\dot V \le 0$ & 3.) $\dot V = 0 \iff \dot x = 0$ 
% 	  \bottomrule
	  \end{tabular}
	 
	\end{block}
	

	
	  

	\begin{alertblock}{\underline{Lemma:} }
	The energy dissipation $\eta$ is bounded from above by $V$.
	\end{alertblock}



	

\end{frame}


%%%%%%%%%%%%%%%%%%%%%%%%%%%%%%%%%%%%%%%%%%%%%%%%%%%%%%%%%%%%%%%%%%%%%%%%%%%%%%%%%%%%%%%%%%%%%%%%%%%%%%%%%%%%%%%%%%%%%%%%%%%%%%%%%%%%%%%%%%%%%%%%%%%%%%



\begin{frame}
    \frametitle{Shortest paths enjoy minimum infrastructure cost} 

	\begin{alertblock}{\underline{Theorem:} Convergence for the continuous model:}
	  As $x(t)$ approaches a stationary point $\eta(t)$ goes to the length of the corresponding shortest path.
	\end{alertblock}

	The proofs for the discretized version follow a similar approach, except the development of $V(t)$ and $\eta(t)$ is tracked in discrete 
        steps in terms of $\epsilon$.

	\begin{alertblock}{\underline{Theorem:} Convergence for the discrete model:}
	    Physarum yields an $\epsilon$-approximation for undirected case in $\mathcal{O}( nL \log(nL) / \epsilon^3)$ iterations.
	\end{alertblock}


	

\end{frame}

%%%%%%%%%%%%%%%%%%%%%%%%%%%%%%%%%%%%%%%%%%%%%%%%%%%%%%%%%%%%%%%%%%%%%%%%%%%%%%%%%%%%%%%%%%%%%%%%%%%%%%%%%%%%%%%%%%%%%%%%%%%%%%%%%%%%%%%%%%%%%%%%%%%%%%

\subsection{Discussion of the model}

\begin{frame}
    \frametitle{Discussion of the model } 


	\begin{alertblock}{\underline{Pros:}}
	 \begin{itemize}
	  \item Captures Physarum's reactive behaviour
	  \item Converges for general graphs under reasonable assumptions
	  \item Treatment of more general network flow problems might be possible
	 \end{itemize}
	\end{alertblock}


	\begin{alertblock}{\underline{Cons:}}
	 \begin{itemize}
	  \item Hopelessly impractical compared to methods like Dijkstra
	  \item Severe oversimplification of real Physarum
	      \begin{itemize}
	       \item does not capture protoplasmic flow behaviour correctly
	       \item does not model thickness/biochemical oscillations
	       \item does not model synchronization
		
	      \end{itemize}

	 \end{itemize}
	\end{alertblock}


\end{frame}

%%%%%%%%%%%%%%%%%%%%%%%%%%%%%%%%%%%%%%%%%%%%%%%%%%%%%%%%%%%%%%%%%%%%%%%%%%%%%%%%%%%%%%%%%%%%%%%%%%%%%%%%%%%%%%%%%%%%%%%%%%%%%%%%%%%%%%%%%%%%%%%%%%%%%%

\subsection{Future directions}

\begin{frame}
    \frametitle{Multiple food sources $\dots$} 

 
\begin{figure}[h]
 
     \begin{center}
   
      \testbox{\includegraphics[angle=0,clip=true,width= 1.6\onethird, trim = 0 0 0 0]<2>{./pics/physarum_tokyo_steps.jpg}}
      \testbox{\includegraphics[angle=0,clip=true,width= 1.6\onethird, trim = 0 0 0 0]<3>{./pics/physarum_tokyo.jpg}}
      
     
     \end{center}

       \tiny{Tero et al. 2010. Rules for Biologically Inspired Adaptive Network Design.}

\end{figure}
   
\end{frame}



% %%%%%%%%%%%%%%%%%%%%%%%%%%%%%%%%%%%%%%%%%%%%%%%%%%%%%%%%%%%%%%%%%%%%%%%%%%%%%%%%%%%%%%%%%%%%%%%%%%%%%%%%%%%%%%%%%%%%%%%%%%%%%%%%%%%%%%%%%%%%%%%%%%%%%%
% 
% \subsection{Future directions}
% 
% \begin{frame}
%     \frametitle{Exploring synchronisation $\dots$} 
% 
% 
%     \begin{alertblock}{\underline{Hypothesis:} Physarum behaves like Kuramoto oscillators  }
% 
%      
% 
%  
%       
%      
%    
% 
% \begin{figure}
% 
% 
% \begin{minipage}{0.45\linewidth}
% 
%     \begin{equation}
% 	\dot \phi_u(t) = w_u + \sum\limits_{v=1}^{V} k_{uv} sin[ \phi_u(t) - \phi_v( t- \tau_{uv} ) ] \nonumber
%       \end{equation}
% 
%     \begin{center}
%       \begin{tabular}{cc}
% % 	   \toprule  
% 	   Coupling $k_{uv}$ & Time delay $\tau_{uv}$ \\
% 	   $\propto x_e$ & $ \propto \frac{1}{v_{wave}} $ \\ 
% 	   $ \propto \frac{1}{l_e} $ & $ \propto l_e $ \\ 
% % 	  \bottomrule
% 	  \end{tabular}
%     \end{center} 
% \end{minipage}
% \hspace{0.84cm}
% \begin{minipage}{0.45\linewidth}
% \centering
% \includegraphics[angle=0,clip=true,width=0.7\linewidth, trim = 275 0 0 0]{./pics/model_new.pdf}
% \end{minipage}
% 
% \end{figure}
% 
%  \end{alertblock}
% 
% 
%     \begin{block}{\underline{Possible approaches:}}
%       \begin{itemize}	      
% 	       \item Fix $l_e$ for a graph. Find $x_e$ s. t. synchronisation is optimal.
% 	       \item Fix $x_e$ for a graph. Find $l_e$ s. t. synchronisation is optimal.	
% 	       \item $\dots$
%       \end{itemize}
%     \end{block}
% 
%    
% \end{frame}


%%%%%%%%%%%%%%%%%%%%%%%%%%%%%%%%%%%%%%%%%%%%%%%%%%%%%%%%%%%%%%%%%%%%%%%%%%%%%%%%%%%%%%%%%%%%%%%%%%%%%%%%%%%%%%%%%%%%%%%%%%%%%%%%%%%%%%%%%%%%%%%%%%%%%%

% \section{The Confinement problem} 
% 
% \begin{frame}
%     \frametitle{The confinement problem revisited} 
% 
% 
%     \emph{\underline{Experimental fact:}} 
%     At high energies hadrons undergo a deconfining phase transition and form a 
%     quark-gluon plasma.
% 
%     \emph{\underline{Question:}} 
%     How can this phenomenon be understood in theory?
% 
% 
%     \begin{block}{Computational physics approach:}
% 
%     
% 
%     \begin{enumerate}
%      \item Choose a theoretical framework $\longrightarrow$ Lattice Gauge Theory.
%      \item Define a useful partition function $\longrightarrow$ $Z_{lat}$.
%      \item Simulate the system $\longrightarrow$ Numerical methods.
%      \item Analyze numerical raw data $\longrightarrow$ Obtain observables.
%      \item Find a meaningful interpretation of the results.
%     \end{enumerate}
% 
%     \end{block}
%     
% %     \emph{Conclusion:} Unless $\mathbf{P} = \mathbf{NP}$, no polynomial time exact solution for the problem exists.
% 
% 
% 
% 
% 
% 
% \end{frame}
% 
% 
% %%%%%%%%%%%%%%%%%%%%%%%%%%%%%%%%%%%%%%%%%%%%%%%%%%%%%%%%%%%%%%%%%%%%%%%%%%%%%%%%%%%%%%%%%%%%%%%%%%%%%%%%%%%%%%%%%%%%%%%%%%%%%%%%%%%%%%%%%%%%%%%%%%%%%%
% 
% \section{Lattice gauge theory in a nutshell} 
% 
% \begin{frame}
%     \frametitle{Lattice Gauge Theory in a nutshell} 
% 
%       \underline{Lattice Gauge Theory}
% 	\begin{itemize}
% 	  \item Space-time is discretized on a \DIM{(3+1)} lattice $\Lambda$.
% 	  \item Each lattice site / link corresponds to a physical field specified by the underlying gauge theory.
% 	  \item A configuration $U$ specifies the values of all fields on the lattice $\longrightarrow$ ``snapshot'' 
% 		of the system.
% 	\end{itemize}  
% 
%       \begin{center}
% 	  \testbox{\includegraphics[angle=0,clip=true,width= 1.9\onethird, trim = 0 0 0 0]{./pics/lattice.png}}
%          \testbox{\includegraphics[angle=0,clip=true,width= 1.2\onethird, trim = 100 46 113 37]{./pics/potts_J=0_7_1.pdf}}
% 	
%       \end{center}
% 
%       
% 
% 
% 
% \end{frame}
% 
% 
% %%%%%%%%%%%%%%%%%%%%%%%%%%%%%%%%%%%%%%%%%%%%%%%%%%%%%%%%%%%%%%%%%%%%%%%%%%%%%%%%%%%%%%%%%%%%%%%%%%%%%%%%%%%%%%%%%%%%%%%%%%%%%%%%%%%%%%%%%%%%%%%%%%%%%%
% 
% \section{The partition function in a nutshell} 
% 
% \begin{frame}
%     \frametitle{The partition function in a nutshell} 
% 
%      
% 
% 
%       \underline{The lattice partition function $Z_{lat}$:}
% 	\begin{equation}
% 	  Z_{lat} = \int \e^{-S_G[U]} \MEAS{U}  \nonumber
% 	\end{equation}  
% 
%         \begin{equation}
% 	   S_{G}[U] = \frac{2N}{g^2} \sum_{n \in \Lambda} \sum_{\mu < \nu} \Re \mbox{tr}  \left[ \hat{1} - U_{\mu \nu}(n) \right] \nonumber
% 	\end{equation}  
% 
% 	\begin{equation}
% 	 \MEAS{U} = \prod_{n \in \Lambda} \prod_{\mu = 1}^{4} \int \mathrm{d}U_{\mu}(n) \nonumber
% 	\end{equation}  
% 
% 	\emph{\underline{Problem:}}	
% 	  Solving this severely multidimensional integral over all possible field configurations is not possible in closed form.
% 	
% 
% 
% 
% \end{frame}
% 
% 
% 
% 
% %%%%%%%%%%%%%%%%%%%%%%%%%%%%%%%%%%%%%%%%%%%%%%%%%%%%%%%%%%%%%%%%%%%%%%%%%%%%%%%%%%%%%
% % \section{Approximation of the Problem} 
% %%%%%%%%%%%%%%%%%%%%%%%%%%%%%%%%%%%%%%%%%%%%%%%%%%%%%%%%%%%%%%%%%%%%%%%%%%%%%%%%%%%%%
% \section{Markov Chain Monte Carlo}
% %%%%%%%%%%%%%%%%%%%%%%%%%%%%%%%%%%%%%%%%%%%%%%%%%%%%%%%%%%%%%%%%%%%%%%%%%%%%%%%%%%%%%%%%%%%%%%%%%%%%%%%%%%%%%%%%%%%%%%%%%%%%%%%%%%%%%%%%%%%%%%%%%%%%%%
% 
% \begin{frame}
%     \frametitle{Markov Chain Monte Carlo} 
% 
%     \underline{Expectation value $\EXP{\hat{O}}$ of an observable:}
% 
%     \begin{equation}
%       \EXP{\hat{O}} = \frac{1}{Z_{lat}} \ \int e^{-S_G[U]} \ O[U] \ \MEAS{U}  \approx \frac{1}{n} \sum_{i=1}^{n} O[U_i] \nonumber
%     \end{equation}
% 
%     \textcolor{red}{ \emph{Key idea:} } $\EXP{\hat{O}}$ can be approximated by computing $\hat{O}[U_i]$ using a series of gauge-field configurations
%     $U_i$ distributed according to the Boltzmann-distribution $e^{-S_G[U]}$.
% 
%     \underline{Markov Chain Monte Carlo (MCMC):}
% 
%     Describes a class of algorithms which exploit stochastic processes to produce a series of samples that
%     mimic a given target probability distribution. 
%     
% 
% 
% 
% \end{frame}
% 
% 
% % \section{Approximation of the Problem} 
% %%%%%%%%%%%%%%%%%%%%%%%%%%%%%%%%%%%%%%%%%%%%%%%%%%%%%%%%%%%%%%%%%%%%%%%%%%%%%%%%%%%%%
% \subsection{Simulation of the system}
% %%%%%%%%%%%%%%%%%%%%%%%%%%%%%%%%%%%%%%%%%%%%%%%%%%%%%%%%%%%%%%%%%%%%%%%%%%%%%%%%%%%%%%%%%%%%%%%%%%%%%%%%%%%%%%%%%%%%%%%%%%%%%%%%%%%%%%%%%%%%%%%%%%%%%%
% 
% \begin{frame}
%     \frametitle{Simulating the system} 
% 
%    
%    
%     \begin{block}{Simulating a Lattice Gauge Theory:}
% 
%       \begin{enumerate}
%        \item Lattice $\longrightarrow$ a plain \DIM{(3+1)} array.
%        \item Set $T$ $\longrightarrow$ determines the initial state the system is in.
%        \item Algorithm $\longrightarrow$ MCMC produces the $U_i$ for calculating $\EXP{\hat{O}}$.
%        \item Repeat steps (2) and (3) to probe the system's behaviour at different $T$.
%       \end{enumerate}
% 
%      
%     \end{block}
% 
% 
%   
%     
% 
% 
% 
% \end{frame}
% 
% 
% % %%%%%%%%%%%%%%%%%%%%%%%%%%%%%%%%%%%%%%%%%%%%%%%%%%%%%%%%%%%%%%%%%%%%%%%%%%%%%%%%%%%%%%%%%%%%%%%%%%%%%%%%%%%%%%%%%%%%%%%%%%%%%%%%%%%%%%%%%%%%%%%%%%%%%%
% % 
% \subsection{MCMC visualized}
% 
% \begin{frame}
%     \frametitle{MCMC visualized } 
% 
% 
%  \vspace{0cm}
% 
% \begin{columns}
% \begin{column}{4.5cm}
% 
%  Sample configurations $U_i$:
% 
% \begin{overprint}
% 
% \begin{itemize}
% 
%   \item<1-> $T = 0.1 T_c $ \\sub-critical phase.
%   \item<4-> $T \approx T_c $ \\critical phase.
%   \item<7-> $T = 1.5 T_c $ \\super-critical phase.
% 
% % \item<2-> Transformation of the problem instance.
% % \item<3-> Partition the instance using a quadtree.
% % \item<4-> Place portals on grid lines.
% % \item<5-> Compute the smallest salesman tour.
% % \item<6-> Trim the edges and output the result.
% \end{itemize}
% % \vspace{3cm} 
% 
% 
% \end{overprint}
% 
% \end{column}
% 
% \begin{column}{4cm}
% \begin{overprint}
% 
% 
% \testbox{
%      \begin{minipage}[t]{5 cm}
% 
% \begin{figure}[h]
%  
%      \begin{center}
%       \testbox{\includegraphics[angle=0,clip=true,width= 1.3\onethird, trim = 114 46 113 37]<1>{./pics/potts_J=0_7_1.pdf}}
%       \testbox{\includegraphics[angle=0,clip=true,width= 1.3\onethird, trim = 114 46 113 37]<2>{./pics/potts_J=0_7_2.pdf}}
%       \testbox{\includegraphics[angle=0,clip=true,width= 1.3\onethird, trim = 114 46 113 37]<3>{./pics/potts_J=0_7_3.pdf}}
%       \testbox{\includegraphics[angle=0,clip=true,width= 1.3\onethird, trim = 114 46 113 37]<4>{./pics/potts_J=1_0_1.pdf}}
%       \testbox{\includegraphics[angle=0,clip=true,width= 1.3\onethird, trim = 114 46 113 37]<5>{./pics/potts_J=1_0_2.pdf}}
%       \testbox{\includegraphics[angle=0,clip=true,width= 1.3\onethird, trim = 114 46 113 37]<6>{./pics/potts_J=1_0_3.pdf}}
%       \testbox{\includegraphics[angle=0,clip=true,width= 1.3\onethird, trim = 114 46 113 37]<7>{./pics/potts_J=1_2_1.pdf}}
%       \testbox{\includegraphics[angle=0,clip=true,width= 1.3\onethird, trim = 114 46 113 37]<8>{./pics/potts_J=1_2_2.pdf}}
%       \testbox{\includegraphics[angle=0,clip=true,width= 1.3\onethird, trim = 114 46 113 37]<9>{./pics/potts_J=1_2_3.pdf}}
%      \end{center}
% 
% \end{figure}
%      \end{minipage} }
% 
% \testbox{
%      \begin{minipage}{5 cm}
%       \begin{center}
% 
% % 	\begin{itemize}
% % 
% 	\item pixel $\widehat{=}$ interacting gauge fields $U(n)$.
% 	
% %     
% % 	\end{itemize}
% 	
% 	
% 
%       \end{center}
%      \end{minipage} }
% 
% 
% 
% 
% 
% 
% \end{overprint}
% \end{column}
% \end{columns}
% 
% 
% \centering
%  \small{Illustrations courtesy of H. Antlinger.}
% 
% 
% 
% \end{frame}
% 
% 
% 
% %%%%%%%%%%%%%%%%%%%%%%%%%%%%%%%%%%%%%%%%%%%%%%%%%%%%%%%%%%%%%%%%%%%%%%%%%%%%%%%%%%%%%%%%%%%%%%%%%%%%%%%%%%%%%%%%%%%%%%%%%%%%%%%%%%%%%%%%%%%%%%%%%%%%%%
% 
% %%%%%%%%%%%%%%%%%%%%%%%%%%%%%%%%%%%%%%%%%%%%%%%%%%%%%%%%%%%%%%%%%%%%%%%%%%%%%%%%%%%%%%%%%%%
% \section{The order parameter of the confinement transition}
% %%%%%%%%%%%%%%%%%%%%%%%%%%%%%%%%%%%%%%%%%%%%%%%%%%%%%%%%%%%%%%%%%%%%%%%%%%%%%%%%%%%%%%%%%%%
% 
% \begin{frame}
%     \frametitle{The order parameter of the confinement transition} 
% 
%    
%     \begin{centering}
%     \begin{figure}
%      
%    
% 
% 
%      \testbox{\includegraphics[angle=90,clip=true,width= 1.7\onehalf, trim = 0 0 0 0]{./pics/results/203040x8_abs_pol.pdf}}
% 
%      \FIGURE{} The order parameter $<|P|>$ as a function of $T/T_c$.
%     
%     \end{figure}
%     \end{centering}
% 
% \end{frame}
% 
% %%%%%%%%%%%%%%%%%%%%%%%%%%%%%%%%%%%%%%%%%%%%%%%%%%%%%%%%%%%%%%%%%%%%%%%%%%%%%%%%%%%%%%%%%%%%%%%%%%%%%%%%%%%%%%%%%%%%%%%%%%%%%%%%%%%%%%%%%%%%%%%%%%%%%%
% 
% %%%%%%%%%%%%%%%%%%%%%%%%%%%%%%%%%%%%%%%%%%%%%%%%%%%%%%%%%%%%%%%%%%%%%%%%%%%%%%%%%%%%%%%%%%%
% \section{A novel approach to confinement: Clustering}
% %%%%%%%%%%%%%%%%%%%%%%%%%%%%%%%%%%%%%%%%%%%%%%%%%%%%%%%%%%%%%%%%%%%%%%%%%%%%%%%%%%%%%%%%%%%
% 
% \begin{frame}
%     \frametitle{A novel approach to confinement} 
% 
% 
% 
% 
% 
%      \begin{centering}
%     \begin{figure}
%      
%    
% 
% 
%      \testbox{\includegraphics[angle=0,clip=true,width= \onethird, trim = 100 46 113 37]{./pics/potts_J=0_7_1.pdf}}
%      \testbox{\includegraphics[angle=0,clip=true,width= \onethird, trim = 100 46 113 37]{./pics/potts_J=1_0_1.pdf}}
%      \testbox{\includegraphics[angle=0,clip=true,width= \onethird, trim = 100 46 113 37]{./pics/potts_J=1_2_1.pdf}}
%       
% 	 
%     \end{figure}
% 
%     \end{centering}
% 
%     \emph{\underline{Question:}} 
%     Is there a correspondence between the behaviour of clusters and the thermal phase transition
%     of confinement?
% 
%     \emph{ \underline{Answer:}} 
%     Under certain conditions a cluster starts to percolate precisely at $T_c$. 
% 
%     \emph{ \underline{Percolation:}} 
%     A cluster is said to be percolating if it connects two adjacent lattice boundaries. 
% 
% 
% 
% \end{frame}
% 
% %%%%%%%%%%%%%%%%%%%%%%%%%%%%%%%%%%%%%%%%%%%%%%%%%%%%%%%%%%%%%%%%%%%%%%%%%%%%%%%%%%%%%%%%%%%%%%%%%%%%%%%%%%%%%%%%%%%%%%%%%%%%%%%%%%%%%%%%%%%%%%%%%%%%%%
% %%%%%%%%%%%%%%%%%%%%%%%%%%%%%%%%%%%%%%%%%%%%%%%%%%%%%%%%%%%%%%%%%%%%%%%%%%%%%%%%%%%%%%%%%%%
% \subsection{The cluster-search problem}
% %%%%%%%%%%%%%%%%%%%%%%%%%%%%%%%%%%%%%%%%%%%%%%%%%%%%%%%%%%%%%%%%%%%%%%%%%%%%%%%%%%%%%%%%%%%
% 
% 
% \begin{frame}
%     \frametitle{Clustering on the lattice} 
% 
% 
%     \underline{Clustering criterion:}
%     A cluster is an agglomeration of nearest-neighbour lattice sites (pixel) with the same phase (color).
% 
% 
% 
% \begin{columns}
% \begin{column}{4.5cm}
% \begin{overprint}
% 
% 
% 
%     \emph{\underline{Questions:}}
%     \begin{itemize}
%      \item Are there percolating clusters?
%      \item How many clusters per phase are there?
%      \item What are their respective sizes?
%      \item \dots
%     \end{itemize}
% 
%     \end{overprint}
% 
% \end{column}
% 
% \begin{column}{5cm}
% \begin{overprint}
% 
% \begin{figure}[h]
%  
%      \begin{center}
%       \testbox{\includegraphics[angle=0,clip=true,width= 1.2\onethird, trim = 114 46 113.5 37]{./pics/potts_J=1_2_2.pdf}}
%      \end{center}
% 
% \end{figure}
%     
% \end{overprint}
% \end{column}
% \end{columns}
% 
% 
%   \emph{\underline{Solution method:}} The Union-Find algorithm.
%   
% 
%     
%     
% \end{frame}
% 
% %%%%%%%%%%%%%%%%%%%%%%%%%%%%%%%%%%%%%%%%%%%%%%%%%%%%%%%%%%%%%%%%%%%%%%%%%%%%%%%%%%%%%%%%%%%%%%%%%%%%%%%%%%%%%%%%%%%%%%%%%%%%%%%%%%%%%%%%%%%%%%%%%%%%%%
% 
% %%%%%%%%%%%%%%%%%%%%%%%%%%%%%%%%%%%%%%%%%%%%%%%%%%%%%%%%%%%%%%%%%%%%%%%%%%%%%%%%%%%%%%%%%%%
% \subsection{The UNION-FIND algorithm}
% %%%%%%%%%%%%%%%%%%%%%%%%%%%%%%%%%%%%%%%%%%%%%%%%%%%%%%%%%%%%%%%%%%%%%%%%%%%%%%%%%%%%%%%%%%%
% 
% \begin{frame}
%     \frametitle{Disjoint-set forests and the Union-Find Algorithm} 
% 
%       \underline{Goal:} Partition a given set (lattice sites) into equivalence classes (clusters) 
%       according to some equivalence relation (clustering criterion).
% 
%       \underline{Disjoint-Tree-Forest:}
%       \begin{itemize}
%        \item Each disjoint set is represented by a tree data structure. 
%        \item All child-nodes hold a reference to their respective parents.
%        \item The root of the tree is the unique representative of the set.
%       \end{itemize}
% 
%       \underline{Supported key operations:}
%       \begin{itemize}
%        \item MAKE-SET($x$): Creates a singleton tree containing $x$.
%        \item FIND($x$).
%        \item UNION($x,y$).
%       \end{itemize}
% 
% \end{frame}
% 
% 
% 
% 
% %%%%%%%%%%%%%%%%%%%%%%%%%%%%%%%%%%%%%%%%%%%%%%%%%%%%%%%%%%%%%%%%%%%%%%%%%%%%%%%%%%%%%%%%%%%%%%%%%%%%%%%%%%%%%%%%%%%%%%%%%%%%%%%%%%%%%%%%%%%%%%%%%%%%%%
%    
% 
% 
% \begin{frame}
%    \frametitle{The UNION and FIND operations} 
% 
% \begin{columns}
% \begin{column}{5cm}
% \begin{overprint}
% 
% 
% 
%     \underline{FIND($x$):}
%     Returns the root node of the tree containing element $x$.
% 
%    
% 
%     \end{overprint}
% 
% \end{column}
% 
% \begin{column}{5cm}
% \begin{overprint}
% 
% \begin{figure}[h]
% 
%      \begin{center}
%       \testbox{\includegraphics[angle=0,clip=true,width= 0.8\onethird, trim = 0 0 0 0]{./pics/AppendixA/rooted_tree2.jpg}}
%      
%       \FIGURE{} FIND($7$) = $3$.
%      \end{center}
% 
% \end{figure}
%     
% \end{overprint}
% \end{column}
% \end{columns}
% %%%%%%%%%%%%%%%%%%%%%%%%%%%%%%%%%%%%
% \begin{columns}
% \begin{column}{5cm}
% \begin{overprint}
% 
% 
% 
%     \underline{UNION($x,y$):}
%     Merges the tree containing element $x$ with the tree containing element $y$.
% 
%    
% 
%     \end{overprint}
% 
% \end{column}
% 
% \begin{column}{5cm}
% \begin{overprint}
% 
% \begin{figure}[h]
% 
%      \begin{center}
%       \testbox{\includegraphics[angle=0,clip=true,width= 2.1\onethird, trim = 0 0 0 0]{./pics/AppendixA/union_operation.jpg}}
%      
%       \FIGURE{} UNION($3,A$).
%      \end{center}
% 
% \end{figure}
%     
% \end{overprint}
% \end{column}
% \end{columns}
%     
% 
% 
% \end{frame}
% 
% %%%%%%%%%%%%%%%%%%%%%%%%%%%%%%%%%%%%%%%%%%%%%%%%%%%%%%%%%%%%%%%%%%%%%%%%%%%%%%%%%%%%%%%%%%%%%%%%%%%%%%%%%%%%%%%%%%%%%%%%%%%%%%%%%%%%%%%%%%%%%%%%%%%%%%
%    
% 
% 
% \begin{frame}
%    \frametitle{Improvements}
% 
%     \underline{Basic idea:} The more balanced and flat the trees are, the more efficient FIND($x$) operates on them.
% 
%     \underline{Heuristics:}
%     \begin{itemize}
%      \item UNION by rank: Lower ranked trees are merged into higher ranked trees.
% % 	  
% % 	  \begin{itemize}
% % 	    \item Rank is unchanged or increased by at most 1.
% % 	    \item Reduces the time FIND takes to return the root-node.
% % 	  \end{itemize}
% 
%      \item Path-compression: All child-nodes traversed by FIND($x$) are redirected to point at the root-node.
% 
% % 	\begin{itemize}
% % 	 \item Reduces the time FIND will take in subsequent calls.
% % 	\end{itemize}
% 
%     \end{itemize}
% 
%     \underline{Amortized costs:}
%     \begin{itemize}
%      \item FIND($x$): $\mathcal{O}( \ln n) $.
%      \item UNION($x,y$): $\mathcal{O}(1)$.
% 
%     \end{itemize}
% 
%   Equipped with these heuristics UNION-FIND takes an amortized $\mathcal{O}(t \, \alpha(t,m))$, which is considered optimal.
%     
% 
% 
% \end{frame}
% 
% 
% % 
% %%%%%%%%%%%%%%%%%%%%%%%%%%%%%%%%%%%%%%%%%%%%%%%%%%%%%%%%%%%%%%%%%%%%%%%%%%%%%%%%%%%%%%%%%%%
% \subsection{Putting the pieces together}
% %%%%%%%%%%%%%%%%%%%%%%%%%%%%%%%%%%%%%%%%%%%%%%%%%%%%%%%%%%%%%%%%%%%%%%%%%%%%%%%%%%%%%%%%%%%
% 
% 
% \begin{frame}
%     \frametitle{ \dots back to the orignal problem } 
%    
%     \underline{Cluster search:}
% 
%       Using the UNION-FIND algorithm the configurations produced in the MCMC runs 
%       can efficiently be processed to obtain expectation values of cluster properties.
% 
%     
%     \underline{Percolation properties:}
% 
%       To check whether or not the confinement transition coincides with the onset of percolation 
%       percolating clusters are identified on the lattice.
% 
% \end{frame}
% % 
% % 
% %%%%%%%%%%%%%%%%%%%%%%%%%%%%%%%%%%%%%%%%%%%%%%%%%%%%%%%%%%%%%%%%%%%%%%%%%%%%%%%%%%%%%%%%%%%
% \section{Confinement and percolation}
% %%%%%%%%%%%%%%%%%%%%%%%%%%%%%%%%%%%%%%%%%%%%%%%%%%%%%%%%%%%%%%%%%%%%%%%%%%%%%%%%%%%%%%%%%%%
% 
% \begin{frame}
%     \frametitle{Confinement and percolation} 
% 
%     
%      
%     \begin{centering}
%     \begin{figure}
%      
%    
% 
%      \testbox{\includegraphics[angle=90,clip=true,width= 1.7\onehalf, trim = 0 0 0 0]{./pics/results/40x6810_percolation_probability.pdf}}
% 
%      \FIGURE{} The percolation probability $<P>$ as a function of $T/T_c$.
%     
%     \end{figure}
%     \end{centering}
% 
%  
%   
% 
% 
% \end{frame}
% % 
% %%%%%%%%%%%%%%%%%%%%%%%%%%%%%%%%%%%%%%%%%%%%%%%%%%%%%%%%%%%%%%%%%%%%%%%%%%%%%%%%%%%%%%%%%%%%%%%%%%%%%%%%%%%%%%%%%%%%%%%%%%%%%%%%%%%%%%%%%%%%%%%%%%%%%%

%%%%%%%%%%%%%%%%%%%%%%%%%%%%%%%%%%%%%%%%%%%%%%%%%%%%%%%%%%%%%%%%%%%%%%%%%%%%%%%%%%%%%%%%%%%
\subsection{Summary and conclusions}
%%%%%%%%%%%%%%%%%%%%%%%%%%%%%%%%%%%%%%%%%%%%%%%%%%%%%%%%%%%%%%%%%%%%%%%%%%%%%%%%%%%%%%%%%%%

\begin{frame}
    \frametitle{Summary and conclusions}   

    \begin{alertblock}{}
    \begin{itemize}
     \item A simple model of the behavior of Physarum Polycephalum can be used to compute shortest paths.
     \item Many open questions regarding possible improvements/extensions suitable for computing $\longrightarrow$ network flows.
     \item Teaser: Model and investigate Physarum's synchronization capabilities based on time delayed oscillators.
     \item Understanding Pysarum challenges and fascinates Computer Scientist, Physicist and Biologist alike.
    \end{itemize}
    \end{alertblock}
 


\end{frame}


%%%%%%%%%%%%%%%%%%%%%%%%%%%%%%%%%%%%%%%%%%%%%%%%%%%%%%%%%%%%%%%%%%%%%%%%%%%%%%%%%%%%%%%%%%%
\subsection{Acknowledgments}
%%%%%%%%%%%%%%%%%%%%%%%%%%%%%%%%%%%%%%%%%%%%%%%%%%%%%%%%%%%%%%%%%%%%%%%%%%%%%%%%%%%%%%%%%%%

\begin{frame}
    \frametitle{Acknowledgments}   

    \begin{figure}
        
       \begin{center}

	    \includegraphics[angle=0,clip=true,width=0.5\textwidth, trim = 0 0 0 0]{./pics/KIST_Logo.jpeg}
	
	   

      \end{center}

      \end{figure}


\end{frame}

%%%%%%%%%%%%%%%%%%%%%%%%%%%%%%%%%%%%%%%%%%%%%%%%%%%%%%%%%%%%%%%%%%%%%%%%%%%%%%%%%%%%%%%%%%%%%%%%%%%%%%%%%%%%%%%%%%%%%%%%%%%%%%%%%%%%%%%%%%%%%%%%%%%%%%
% 
% 
% %%%%%%%%%%%%%%%%%%%%%%%%%%%%%%%%%%%%%%%%%%%%%%%%%%%%%%%%%%%%%%%%%%%%%%%%%%%%%%%%%%%%%%%%%%%%%%%%%%%%%%%%%%%%%%%%%%%%%%%%%%%%%%%%%%%%%%%%%%%%%%%%%%%%%%
% \section{Sketch of Proof} 
% %%%%%%%%%%%%%%%%%%%%%%%%%%%%%%%%%%%%%%%%%%%%%%%%%%%%%%%%%%%%%%%%%%%%%%%%%%%%%%%%%%%%%%%%%%%%%%%%%%%%%%%%%%%%%%%%%%%%%%%%%%%%%%%%%%%%%%%%%%%%%%%%%%%%%%
% 
% %%%%%%%%%%%%%%%%%%%%%%%%%%%%%%%%%%%%%%%%%%%%%%%%%%%%%%%%%%%%%%%%%%%%%%%%%%%%%%%%%%%%%%%%%%%%%%%%%%%%%%%%%%%%%%%%%%%%%%%%%%%%%%%%%%%%%%%%%%%%%%%%%%%%%%
% 
% \begin{frame}
%     \frametitle{Correctness of the PTAS} 
% 
% %     The correctness of the PTAS relies on the following theorem.
% 
%     \begin{block}{Structure Theorem}
%      Given an $(a, b)$-shifted dissection,
%      there exists a $(1 + \frac{1}{c})$-approximate 
%      salesman tour with a probability of at least $1/2$, if it crosses the boundary of each square in the shifted quadtree at most $2$ times and does so 
%       always at portals.
% 
%     \end{block}
% 
%     The proof of the Structure Theorem is intricate, but the basic ingredients 
%     follow from the truth of the following statements:
%     
%     \begin{itemize}
%      \item It is not too expensive to ``bend'' the optimal tour into a portal-respecting tour (Portal-respecting Lemma).
%      \item It is not too expensive to convert a portal-respecting tour into an $2$-light portal respecting tour (Patching Lemma).
%     \end{itemize}
% 
% 
% 
% 
% 
% 
% \end{frame}
% 
% \section{Some Practical Considerations} 
% 
% \begin{frame}
%  
% \frametitle{Some Practical Considerations}
% 
%  
% 
% 	  \emph{Pro features:}
% 	  \begin{itemize}
% 	   \item Can be combined with recent ``branch-and-cut'' methods to boost performance.
% 	   \item Can be parallelized.
% 	  \end{itemize}
% 
% 
% 	  \emph{Contra features:}
% 	  \begin{itemize}
% 	   \item Cannot compete with other heuristics for large problem instances.
% 	   \item Runtime depends exponentially on $c$ even in $\mathbb{R}^2$.
% 	  \end{itemize}
% 
%         
% 	  \emph{Remark:} A comparison of Arora's PTAS and the Concorde TSP solver can be found here \cite{Rodecker}.
% 	  
% 
% \end{frame}
% 
% 
% %%%%%%%%%%%%%%%%%%%%%%%%%%%%%%%%%%%%%%%%%%%%%%%%%%%%%%%%%%%%%%%%%%%%%%%%%%%%%%%%%%%%%%%%%%%%%%%%%%%%%%%%%%%%%%%%%%%%%%%%%%%%%%%%%%%%%%%%%%%%%%%%%%%%%%
% 
% %%%%%%%%%%%%%%%%%%%%%%%%%%%%%%%%%%%%%%%%%%%%%%%%%%%%%%%%%%%%%%%%%%%%%%%%%%%%%%%%%%%%%%%%%%%%%%%%%%%%%%%%%%%%%%%%%%%%%%%%%%%%%%%%%%%%%%%%%%%%%%%%%%%%%%
% \section{Central ideas of Arora's PTAS Approach} 
% %%%%%%%%%%%%%%%%%%%%%%%%%%%%%%%%%%%%%%%%%%%%%%%%%%%%%%%%%%%%%%%%%%%%%%%%%%%%%%%%%%%%%%%%%%%%%%%%%%%%%%%%%%%%%%%%%%%%%%%%%%%%%%%%%%%%%%%%%%%%%%%%%%%%%%
% 
% 
% \begin{frame}
%     \frametitle{Summary} 
% 
%     \begin{block}{Central ideas of Arora's PTAS}
%      \begin{itemize}
%       \item Map the given problem instance to a coarser instance, depending on the error parameter.
%       \item Apply some kind of geometric partition to subdivide the problem.
%       \item Apply suitable restrictions and exploit the partition to rephrase the problem as a family of near-independent subproblems.
%       \item Obtain an approximate solution by means of dynamic programming, combining sub-solutions in a ``divide and conquer''
%       fashion.
%      \end{itemize}
% 
%     \end{block}
% 
% %  The design of almost all known FPTAS’s and PTAS’s is based on the idea
% % of trading accuracy for running time – the given problem instance is mapped
% % to a coarser instance, depending on the error parameter ε, which is solved
% % optimally by a dynamic programming approach. The latter ends up being an
% % exhaustive search of polynomially many different possibilities (for instance,
% % for knapsack, this involves computing A(i, p) for all i and p). In most such
% % algorithms, the running time is prohibitive even for reasonable n and ε. Fur-
% % ther, if the algorithm had to resort to exhaustive search, does the problem
% % really offer “footholds” to home in on a solution efficiently? Is an FPTAS or
% % PTAS the best one can hope for for an NP-hard problem? Clearly, the issue
% % is complex and there is no straightforward answer.
% 
% 
% \end{frame}
% 
% 
% 
% 
% 
% 
% %%%%%%%%%%%%%%%%%%%%%%%%%%%%%%%%%%%%%%%%%%%%%%%%%%%%%%%%%%%%%%%%%%%%%%%%%%%%%%%%%%%%%%%%%%%%%%%%%%%%%%%%%%%%%%%%%%%%%%%%%%%%%%%%%%%%%%%%%%%%%%%%
% 
% \begin{frame}[allowframebreaks]
% \frametitle{References} 
% 
%  \bibliographystyle{springerrefs}
%     \bibliography{papers}
% 
% \end{frame}
% 
% \begin{frame}[plain]
%  
% \end{frame}
% 
% % 
% % %%%%%%%%%%%%%%%%%%%%%%%%%%%%%%%%%%%%%%%%%%%%%%%%%%%%%%%%%%%%%%%%%%%%%%%%%%%%%%%%%%%%%%%%%%%%%%%%%%%%%%%%%%%%%%%%%%%%%%%%%%%%%%%%%%%%%%%%%%%%%%%%
% % \appendix
% % %%%%%%%%%%%%%%%%%%%%%%%%%%%%%%%%%%%%%%%%%%%%%%%%%%%%%%%%%%%%%%%%%%%%%%%%%%%%%%%%%%%%%%%%%%%%%%%%%%%%%%%%%%%%%%%%%%%%%%%%%%%%%%%%%%%%%%%%%%%%%%%%
% % 
% % %%%%%%%%%%%%%%%%%%%%%%%%%%%%%%%%%%%%%%%%%%%%%%%%%%%%%%%%%%%%%%%%%%%%%%%%%%%%%%%%%%%%%%%%%%%%%%%%%%%%%%%%%%%%%%%%%%%%%%%%%%%%%%%%%%%%%%%%%%%%%%%%
% % 
% % \section{Scatter-plot at Tc}
% % \begin{frame}
% %     \frametitle{Spatially Averaged Polyakov Loop at $T_c$} 
% % 
% % 
% % \end{frame}
% % 
% % %%%%%%%%%%%%%%%%%%%%%%%%%%%%%%%%%%%%%%%%%%%%%%%%%%%%%%%%%%%%%%%%%%%%%%%%%%%%%%%%%%%%%%%%%%%%%%%%%%%%%%%%%%%%%%%%%%%%%%%%%%%%%%%%%%%%%%%%%%%%%%%%
% % \section{Size of the Largest Cluster}
% % %%%%%%%%%%%%%%%%%%%%%%%%%%%%%%%%%%%%%%%%%%%%%%%%%%%%%%%%%%%%%%%%%%%%%%%%%%%%%%%%%%%%%%%%%%%%%%%%%%%%%%%%%%%%%%%%%%%%%%%%%%%%%%%%%%%%%%%%%%%%%%%%
% % 
% % \begin{frame}
% %     \frametitle{Size of the Largest Cluster} 
% % 
% %           
% % \end{frame}
% % 
% 
\end{document}
                                       
